% Options for packages loaded elsewhere
\PassOptionsToPackage{unicode}{hyperref}
\PassOptionsToPackage{hyphens}{url}
%
\documentclass[
]{article}
\usepackage{lmodern}
\usepackage{amssymb,amsmath}
\usepackage{ifxetex,ifluatex}
\ifnum 0\ifxetex 1\fi\ifluatex 1\fi=0 % if pdftex
  \usepackage[T1]{fontenc}
  \usepackage[utf8]{inputenc}
  \usepackage{textcomp} % provide euro and other symbols
\else % if luatex or xetex
  \usepackage{unicode-math}
  \defaultfontfeatures{Scale=MatchLowercase}
  \defaultfontfeatures[\rmfamily]{Ligatures=TeX,Scale=1}
\fi
% Use upquote if available, for straight quotes in verbatim environments
\IfFileExists{upquote.sty}{\usepackage{upquote}}{}
\IfFileExists{microtype.sty}{% use microtype if available
  \usepackage[]{microtype}
  \UseMicrotypeSet[protrusion]{basicmath} % disable protrusion for tt fonts
}{}
\makeatletter
\@ifundefined{KOMAClassName}{% if non-KOMA class
  \IfFileExists{parskip.sty}{%
    \usepackage{parskip}
  }{% else
    \setlength{\parindent}{0pt}
    \setlength{\parskip}{6pt plus 2pt minus 1pt}}
}{% if KOMA class
  \KOMAoptions{parskip=half}}
\makeatother
\usepackage{xcolor}
\IfFileExists{xurl.sty}{\usepackage{xurl}}{} % add URL line breaks if available
\IfFileExists{bookmark.sty}{\usepackage{bookmark}}{\usepackage{hyperref}}
\hypersetup{
  pdftitle={邹易+202111106+lab1},
  pdfauthor={zouyi},
  hidelinks,
  pdfcreator={LaTeX via pandoc}}
\urlstyle{same} % disable monospaced font for URLs
\usepackage[margin=1in]{geometry}
\usepackage{color}
\usepackage{fancyvrb}
\newcommand{\VerbBar}{|}
\newcommand{\VERB}{\Verb[commandchars=\\\{\}]}
\DefineVerbatimEnvironment{Highlighting}{Verbatim}{commandchars=\\\{\}}
% Add ',fontsize=\small' for more characters per line
\usepackage{framed}
\definecolor{shadecolor}{RGB}{248,248,248}
\newenvironment{Shaded}{\begin{snugshade}}{\end{snugshade}}
\newcommand{\AlertTok}[1]{\textcolor[rgb]{0.94,0.16,0.16}{#1}}
\newcommand{\AnnotationTok}[1]{\textcolor[rgb]{0.56,0.35,0.01}{\textbf{\textit{#1}}}}
\newcommand{\AttributeTok}[1]{\textcolor[rgb]{0.77,0.63,0.00}{#1}}
\newcommand{\BaseNTok}[1]{\textcolor[rgb]{0.00,0.00,0.81}{#1}}
\newcommand{\BuiltInTok}[1]{#1}
\newcommand{\CharTok}[1]{\textcolor[rgb]{0.31,0.60,0.02}{#1}}
\newcommand{\CommentTok}[1]{\textcolor[rgb]{0.56,0.35,0.01}{\textit{#1}}}
\newcommand{\CommentVarTok}[1]{\textcolor[rgb]{0.56,0.35,0.01}{\textbf{\textit{#1}}}}
\newcommand{\ConstantTok}[1]{\textcolor[rgb]{0.00,0.00,0.00}{#1}}
\newcommand{\ControlFlowTok}[1]{\textcolor[rgb]{0.13,0.29,0.53}{\textbf{#1}}}
\newcommand{\DataTypeTok}[1]{\textcolor[rgb]{0.13,0.29,0.53}{#1}}
\newcommand{\DecValTok}[1]{\textcolor[rgb]{0.00,0.00,0.81}{#1}}
\newcommand{\DocumentationTok}[1]{\textcolor[rgb]{0.56,0.35,0.01}{\textbf{\textit{#1}}}}
\newcommand{\ErrorTok}[1]{\textcolor[rgb]{0.64,0.00,0.00}{\textbf{#1}}}
\newcommand{\ExtensionTok}[1]{#1}
\newcommand{\FloatTok}[1]{\textcolor[rgb]{0.00,0.00,0.81}{#1}}
\newcommand{\FunctionTok}[1]{\textcolor[rgb]{0.00,0.00,0.00}{#1}}
\newcommand{\ImportTok}[1]{#1}
\newcommand{\InformationTok}[1]{\textcolor[rgb]{0.56,0.35,0.01}{\textbf{\textit{#1}}}}
\newcommand{\KeywordTok}[1]{\textcolor[rgb]{0.13,0.29,0.53}{\textbf{#1}}}
\newcommand{\NormalTok}[1]{#1}
\newcommand{\OperatorTok}[1]{\textcolor[rgb]{0.81,0.36,0.00}{\textbf{#1}}}
\newcommand{\OtherTok}[1]{\textcolor[rgb]{0.56,0.35,0.01}{#1}}
\newcommand{\PreprocessorTok}[1]{\textcolor[rgb]{0.56,0.35,0.01}{\textit{#1}}}
\newcommand{\RegionMarkerTok}[1]{#1}
\newcommand{\SpecialCharTok}[1]{\textcolor[rgb]{0.00,0.00,0.00}{#1}}
\newcommand{\SpecialStringTok}[1]{\textcolor[rgb]{0.31,0.60,0.02}{#1}}
\newcommand{\StringTok}[1]{\textcolor[rgb]{0.31,0.60,0.02}{#1}}
\newcommand{\VariableTok}[1]{\textcolor[rgb]{0.00,0.00,0.00}{#1}}
\newcommand{\VerbatimStringTok}[1]{\textcolor[rgb]{0.31,0.60,0.02}{#1}}
\newcommand{\WarningTok}[1]{\textcolor[rgb]{0.56,0.35,0.01}{\textbf{\textit{#1}}}}
\usepackage{graphicx,grffile}
\makeatletter
\def\maxwidth{\ifdim\Gin@nat@width>\linewidth\linewidth\else\Gin@nat@width\fi}
\def\maxheight{\ifdim\Gin@nat@height>\textheight\textheight\else\Gin@nat@height\fi}
\makeatother
% Scale images if necessary, so that they will not overflow the page
% margins by default, and it is still possible to overwrite the defaults
% using explicit options in \includegraphics[width, height, ...]{}
\setkeys{Gin}{width=\maxwidth,height=\maxheight,keepaspectratio}
% Set default figure placement to htbp
\makeatletter
\def\fps@figure{htbp}
\makeatother
\setlength{\emergencystretch}{3em} % prevent overfull lines
\providecommand{\tightlist}{%
  \setlength{\itemsep}{0pt}\setlength{\parskip}{0pt}}
\setcounter{secnumdepth}{-\maxdimen} % remove section numbering

\title{邹易+202111106+lab1}
\author{zouyi}
\date{2020/9/30}

\begin{document}
\maketitle

\#设置全局路径

\#work1

\begin{Shaded}
\begin{Highlighting}[]
\NormalTok{cor.data <-}\StringTok{ }\KeywordTok{read.csv}\NormalTok{(}\StringTok{'correlation_data.csv'}\NormalTok{, }\DataTypeTok{header =}\NormalTok{ T)}
\NormalTok{cor.data <-}\StringTok{ }\NormalTok{cor.data[,}\OperatorTok{-}\DecValTok{1}\NormalTok{]  }\CommentTok{#去掉第一列}

\CommentTok{# 计算pearson相关系数}
\CommentTok{# corr.a <- cor(a.x,a.y)}
\CommentTok{# corr.b <- cor(b.x,b.y)}
\CommentTok{# corr.c <- cor(c.x,c.y)}
\CommentTok{# corr.d <- cor(d.x,d.y)}
\CommentTok{# corr.e <- cor(e.x,e.y)}
\CommentTok{# corr.f <- cor(f.x,f.y)}
\CommentTok{# corr.g <- cor(g.x,g.y)}
\CommentTok{# corr.h <- cor(h.x,h.y)}
\NormalTok{corr.pearson <-}\StringTok{ }\KeywordTok{vector}\NormalTok{(}\DataTypeTok{mode =} \StringTok{"numeric"}\NormalTok{, }\DataTypeTok{length =} \DecValTok{8}\NormalTok{)}
\ControlFlowTok{for}\NormalTok{ (i }\ControlFlowTok{in} \KeywordTok{seq}\NormalTok{(}\DecValTok{2}\NormalTok{,}\DecValTok{16}\NormalTok{,}\DecValTok{2}\NormalTok{)) \{}
\NormalTok{  corr.pearson[i}\OperatorTok{/}\DecValTok{2}\NormalTok{] <-}\StringTok{ }\KeywordTok{cor}\NormalTok{(cor.data[,i], cor.data[,i}\DecValTok{-1}\NormalTok{])}
\NormalTok{\}}
\KeywordTok{cat}\NormalTok{(}\StringTok{"pearson correlation:"}\NormalTok{, }\StringTok{'}\CharTok{\textbackslash{}n}\StringTok{'}\NormalTok{, letters[}\DecValTok{1}\OperatorTok{:}\DecValTok{8}\NormalTok{], }\StringTok{'}\CharTok{\textbackslash{}n}\StringTok{'}\NormalTok{, corr.pearson, }\StringTok{'}\CharTok{\textbackslash{}n}\StringTok{'}\NormalTok{)}
\end{Highlighting}
\end{Shaded}

\begin{verbatim}
## pearson correlation: 
##  a b c d e f g h 
##  0.8350736 0.3130239 0.01134618 -0.3201501 -0.004464113 -0.05662948 0.0298095 -0.09294615
\end{verbatim}

\begin{Shaded}
\begin{Highlighting}[]
\CommentTok{#计算Spearman 秩相关系数}
\NormalTok{corr.Spearson <-}\StringTok{ }\KeywordTok{vector}\NormalTok{(}\DataTypeTok{mode =} \StringTok{"numeric"}\NormalTok{, }\DataTypeTok{length =} \DecValTok{8}\NormalTok{)}
\ControlFlowTok{for}\NormalTok{ (i }\ControlFlowTok{in} \KeywordTok{seq}\NormalTok{(}\DecValTok{2}\NormalTok{,}\DecValTok{16}\NormalTok{,}\DecValTok{2}\NormalTok{)) \{}
\NormalTok{  corr.Spearson[i}\OperatorTok{/}\DecValTok{2}\NormalTok{] <-}\StringTok{ }\KeywordTok{cor}\NormalTok{(cor.data[,i], cor.data[,i}\DecValTok{-1}\NormalTok{], }\DataTypeTok{method =} \StringTok{"spearman"}\NormalTok{)}
\NormalTok{\}}
\KeywordTok{cat}\NormalTok{(}\StringTok{"Spearson correlation:"}\NormalTok{, }\StringTok{'}\CharTok{\textbackslash{}n}\StringTok{'}\NormalTok{, letters[}\DecValTok{1}\OperatorTok{:}\DecValTok{8}\NormalTok{], }\StringTok{'}\CharTok{\textbackslash{}n}\StringTok{'}\NormalTok{, corr.Spearson, }\StringTok{'}\CharTok{\textbackslash{}n}\StringTok{'}\NormalTok{)}
\end{Highlighting}
\end{Shaded}

\begin{verbatim}
## Spearson correlation: 
##  a b c d e f g h 
##  0.8277667 0.3203891 0.02894975 -0.348998 0.02100691 -0.06771639 -0.008803729 -0.08317638
\end{verbatim}

\begin{Shaded}
\begin{Highlighting}[]
\NormalTok{spearson.corr <-}\StringTok{ }\ControlFlowTok{function}\NormalTok{(x,y)\{}
  \ControlFlowTok{if}\NormalTok{(}\KeywordTok{length}\NormalTok{(x) }\OperatorTok{!=}\StringTok{ }\KeywordTok{length}\NormalTok{(y))\{}
    \KeywordTok{return}\NormalTok{(}\StringTok{"The lengths of x and y are not equal"}\NormalTok{)}
\NormalTok{  \}}
\NormalTok{  x.rank <-}\StringTok{ }\KeywordTok{rank}\NormalTok{(x)}
\NormalTok{  y.rank <-}\StringTok{ }\KeywordTok{rank}\NormalTok{(y)}
\NormalTok{  n <-}\StringTok{ }\KeywordTok{length}\NormalTok{(x)}
\NormalTok{  temp <-}\StringTok{ }\ControlFlowTok{function}\NormalTok{(r1,r2)\{}
\NormalTok{    n <-}\StringTok{ }\KeywordTok{length}\NormalTok{(r1)}
\NormalTok{    re <-}\StringTok{ }\KeywordTok{sum}\NormalTok{(r1 }\OperatorTok{*}\StringTok{ }\NormalTok{r2)}\OperatorTok{-}\NormalTok{n }\OperatorTok{*}\StringTok{ }\NormalTok{((n }\OperatorTok{+}\StringTok{ }\DecValTok{1}\NormalTok{) }\OperatorTok{/}\StringTok{ }\DecValTok{2}\NormalTok{) }\OperatorTok{^}\StringTok{ }\DecValTok{2}
    \KeywordTok{return}\NormalTok{(re)}
\NormalTok{  \}}
  \CommentTok{# 这是带结的计算公式}
\NormalTok{  corr1 <-}\StringTok{ }\KeywordTok{temp}\NormalTok{(x.rank,y.rank) }\OperatorTok{/}\StringTok{ }\NormalTok{(}\KeywordTok{sqrt}\NormalTok{(}\KeywordTok{temp}\NormalTok{(x.rank,x.rank)) }\OperatorTok{*}\StringTok{ }\KeywordTok{sqrt}\NormalTok{(}\KeywordTok{temp}\NormalTok{(y.rank,y.rank)))}
  \CommentTok{#这是不带结的计算公式}
\NormalTok{  corr2 <-}\StringTok{ }\DecValTok{1} \OperatorTok{-}\StringTok{ }\DecValTok{6} \OperatorTok{*}\StringTok{ }\KeywordTok{sum}\NormalTok{((x.rank }\OperatorTok{-}\StringTok{ }\NormalTok{y.rank) }\OperatorTok{^}\StringTok{ }\DecValTok{2}\NormalTok{) }\OperatorTok{/}\StringTok{ }\NormalTok{(n}\OperatorTok{^}\DecValTok{3}\OperatorTok{-}\NormalTok{n)  }
  \KeywordTok{return}\NormalTok{(}\KeywordTok{c}\NormalTok{(corr1,corr2))}
\NormalTok{\}}
\KeywordTok{spearson.corr}\NormalTok{(cor.data}\OperatorTok{$}\NormalTok{a.x, cor.data}\OperatorTok{$}\NormalTok{a.y) }\CommentTok{#用所编写的函数计算a的Spearson}
\end{Highlighting}
\end{Shaded}

\begin{verbatim}
## [1] 0.8277667 0.8277743
\end{verbatim}

\begin{Shaded}
\begin{Highlighting}[]
\CommentTok{#计算Kendall 秩相关系数}
\NormalTok{corr.kendall <-}\StringTok{ }\KeywordTok{vector}\NormalTok{(}\DataTypeTok{mode =} \StringTok{"numeric"}\NormalTok{, }\DataTypeTok{length =} \DecValTok{8}\NormalTok{)}
\ControlFlowTok{for}\NormalTok{ (i }\ControlFlowTok{in} \KeywordTok{seq}\NormalTok{(}\DecValTok{2}\NormalTok{,}\DecValTok{16}\NormalTok{,}\DecValTok{2}\NormalTok{)) \{}
\NormalTok{  corr.kendall[i}\OperatorTok{/}\DecValTok{2}\NormalTok{] <-}\StringTok{ }\KeywordTok{cor}\NormalTok{(cor.data[,i}\DecValTok{-1}\NormalTok{], cor.data[,i], }\DataTypeTok{method =} \StringTok{"kendall"}\NormalTok{)}
\NormalTok{\}}
\KeywordTok{cat}\NormalTok{(}\StringTok{"kendall correlation:"}\NormalTok{, }\StringTok{'}\CharTok{\textbackslash{}n}\StringTok{'}\NormalTok{, letters[}\DecValTok{1}\OperatorTok{:}\DecValTok{8}\NormalTok{], }\StringTok{'}\CharTok{\textbackslash{}n}\StringTok{'}\NormalTok{, corr.kendall, }\StringTok{'}\CharTok{\textbackslash{}n}\StringTok{'}\NormalTok{)}
\end{Highlighting}
\end{Shaded}

\begin{verbatim}
## kendall correlation: 
##  a b c d e f g h 
##  0.6332125 0.2171152 0.01729049 -0.2375529 0.02719716 -0.04058556 -0.01307256 -0.03327443
\end{verbatim}

\begin{Shaded}
\begin{Highlighting}[]
\NormalTok{kendall.corr <-}\StringTok{ }\ControlFlowTok{function}\NormalTok{(x,y)\{}
  \ControlFlowTok{if}\NormalTok{(}\KeywordTok{length}\NormalTok{(x) }\OperatorTok{!=}\StringTok{ }\KeywordTok{length}\NormalTok{(y))\{}
    \KeywordTok{return}\NormalTok{(}\StringTok{"The lengths of x and y are not equal"}\NormalTok{)}
\NormalTok{  \}}
\NormalTok{  n <-}\StringTok{ }\KeywordTok{length}\NormalTok{(x)}
\NormalTok{  nc =}\StringTok{ }\NormalTok{nd =}\StringTok{ }\DecValTok{0}
  \ControlFlowTok{for}\NormalTok{ (i }\ControlFlowTok{in} \DecValTok{1}\OperatorTok{:}\NormalTok{(n}\DecValTok{-1}\NormalTok{)) \{}
    \ControlFlowTok{for}\NormalTok{ (j }\ControlFlowTok{in}\NormalTok{ (i}\OperatorTok{+}\DecValTok{1}\NormalTok{)}\OperatorTok{:}\NormalTok{n) \{}
\NormalTok{      temp =}\StringTok{ }\NormalTok{(x[j] }\OperatorTok{-}\StringTok{ }\NormalTok{x[i]) }\OperatorTok{*}\StringTok{ }\NormalTok{(y[j] }\OperatorTok{-}\StringTok{ }\NormalTok{y[i])}
      \ControlFlowTok{if}\NormalTok{(temp }\OperatorTok{>}\StringTok{ }\DecValTok{0}\NormalTok{)\{}
\NormalTok{        nc =}\StringTok{ }\NormalTok{nc }\OperatorTok{+}\StringTok{ }\DecValTok{1}
\NormalTok{      \}}
      \ControlFlowTok{if}\NormalTok{(temp }\OperatorTok{<}\StringTok{ }\DecValTok{0}\NormalTok{)\{}
\NormalTok{        nd =}\StringTok{ }\NormalTok{nd }\OperatorTok{+}\StringTok{ }\DecValTok{1}
\NormalTok{      \}}
\NormalTok{    \}}
\NormalTok{  \}}
\NormalTok{  cor1 <-}\StringTok{ }\NormalTok{(nc }\OperatorTok{-}\StringTok{ }\NormalTok{nd) }\OperatorTok{/}\StringTok{ }\KeywordTok{choose}\NormalTok{(n, }\DecValTok{2}\NormalTok{)  }\CommentTok{# 无结}
\NormalTok{  cor2 <-}\StringTok{ }\NormalTok{(nc }\OperatorTok{-}\StringTok{ }\NormalTok{nd) }\OperatorTok{/}\StringTok{ }\NormalTok{(nc }\OperatorTok{+}\StringTok{ }\NormalTok{nd)  }\CommentTok{# 有结 }
  \KeywordTok{return}\NormalTok{(}\KeywordTok{c}\NormalTok{(cor1,cor2))}
\NormalTok{\}}

\KeywordTok{kendall.corr}\NormalTok{(cor.data}\OperatorTok{$}\NormalTok{a.x, cor.data}\OperatorTok{$}\NormalTok{a.y)}
\end{Highlighting}
\end{Shaded}

\begin{verbatim}
## [1] 0.6315578 0.6348757
\end{verbatim}

\end{document}
